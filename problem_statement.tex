\documentclass[12pt]{article}
\usepackage{math}
\begin{document}
\begin{center}
  {\Large MS\&E 408: Independent Study on Generative Flow Networks}

	\begin{tabular}{rl}
		SUNet ID:      & djenson                            \\
		Name:          & Daniel Jenson                      \\
		Advisors: & Prof. Jose Blanchet, Mo Tiwari
	\end{tabular}
\end{center}

\section*{2022-10-14: Session Recap}
Test

\section*{Problem Statement}
Rare event simulation is a difficult task for traditional MCMC methods. In the
class of committer problems, one attempts to measure the probability that a
sequence of events will touch a rare subset. A simple example of this is the
Gambler's Ruin problem with negative drift, i.e. the probability that the
player will win everything given that the probability of winning each bet is $0
< \Pr(\text{Win}) \ll 0.5$.
\par
Vanilla Monte Carlo methods solve for this probability by simulating
trajectories according to the probability of success on each bet and then
accept a sample if it touches the rare subset. For extremely rare events, the
variance is large relative to the estimate since so few events touch the rate
subset.
\par
One variance reducing technique that attempts to improve the sample acceptance
rate is importance sampling with exponential tilting. This method shifts the
distribution toward the rare event but has the convenience of a concise
likelihood ratio or importance sampling ratio for common distributions. A
$\theta$-tilted distribution has the following form:
\[
  \begin{aligned}
      \Pr(X\in\dd x;\theta) &= \frac{\mathbb{E}\left[e^{\theta X}\mathbb{I}[X\in\dd x]\right]}{M_X(\theta)}=e^{\theta X-\kappa(\theta)}\Pr(X\in\dd x)
  \end{aligned}
\]
Where $M_X$ is the moment generating function and $\kappa(\theta)$ is the
cumulant generating function (CGF):
\[
  \begin{aligned}
    \kappa(\theta) &= \log \mathbb{E}\left[e^{\theta X}\right]=\log M_X(\theta) \\
  \end{aligned}
\]
Members of the exponential family often have $\theta$-tilted distributions of
the same form as their original distribution. For example, the normal
distribution becomes
$\operatorname{Normal}\left(\mu+\theta\sigma^2,\sigma^2\right)$ and the binomial
becomes $\operatorname{Binomial}\left(n,\frac{pe^\theta}{1-p+pe^\theta}\right)$.
\par
For a distribution to be exponentially tilted, it must have a probability
measure over the random variable as well as a moment generating function.

\newpage
\subsection*{Current State}
\subsubsection*{Exponential Tilting}
Unfortunately, I'm having trouble defining a probability measure for the
Gambler's Ruin trajectory space. Some common definitions:
\[
  \begin{aligned}
    \mathcal{T}
    &:\text{The set of all complete trajectories}
    \\ \mathcal{T} &=\{\tau = \{X_i\}_{i\in\mathbb{N}}: \tau\in \{A,B\}\}
    \\ X_i
    &:\text{A single bet}
    \\X_i &\sim 2\cdot(\operatorname{Bernoulli}\left(p\right)-0.5)
    \\X_i &\in \{-1, 1\} 
    \\ 0 < p \ll 0.5
    &=\text{probability of winning a bet}
    \\ A
    &:\text{The event that the gambler wins}
    \\ B
    &:\text{The event that the gambler loses}
  \end{aligned}
\]
In order to calculate the exponentially tilted distribution for Gambler's Ruin,
I believe we need both $\Pr(\tau)$ as well as $M_\tau(\theta)$. Since this
problem has boundaries, we cannot use the standard binomial. In other words, if
$\tau$ is a complete trajectory of length $n$, then $\{X_i\}_{i\in 1:n-1}$ must
not have put the trajectory in set $A$ or set $B$. 
\par
There seems to be a similar problem for GFNs, i.e. how do we calculate the
importance sampling weight or what is the relationship between the reward
function and the likelihood?

\end{document}
